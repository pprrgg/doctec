

\newcommand{\MostrarVariablesAlFinal}{}
%\documentclass[a4paper,10pt]{article}
\documentclass[a4paper,10pt,twocolumn]{article}
\usepackage[top = 2cm, bottom = 2cm, left = 2cm, right = 2cm, asymmetric]{geometry} % Especificar los márgenes según la norma
\usepackage[spanish]{babel}    

\usepackage{graphicx}  % Para incluir imágenes si es necesario
\usepackage{amsmath, amssymb}  % Para expresiones matemáticas
\usepackage{fancyhdr}  % Para personalizar encabezados
\usepackage{chngcntr}  % Para cambiar la numeración de apartados y subsecciones
\usepackage{tocbibind}  % Para incluir bibliografía en la tabla de contenidos
\usepackage{appendix}  % Para el formato de anexos
\usepackage{lipsum}  % Para generar texto de relleno (dummy text)
\usepackage{geometry}  % Para personalizar los márgenes
\usepackage{multicol}  % Para columnas
\usepackage{titlesec}  % Para personalizar los títulos de las secciones
\usepackage[utf8]{inputenc}    
\usepackage[T1]{fontenc}       
\usepackage[pdftex,pdfencoding=auto, colorlinks=true, linkcolor=black, citecolor=black, filecolor=black, urlcolor=black]{hyperref}
\usepackage{tocloft}           
\usepackage{booktabs}          
\usepackage{float} % Paquete necesario para usar la opción [H]
\usepackage{array}
\usepackage{longtable}
\usepackage{circuitikz}
\usepackage{tikz}
\usepackage{tikz-cd}
\usepackage{tikz-qtree}
\usepackage{smartdiagram}

\usepackage{qrcode}            % Paquete para generar QR
\usepackage{tabularx} % Agregar en el preámbulo
\usepackage[absolute,overlay]{textpos} % Para posicionar objetos libremente
\usepackage{underscore}
\usepackage{pdfpages}

\pagestyle{fancy}
\fancyhf{}
\fancyhead[L]{OPTIMIZACIÓN DE POTENCIA}  % Nombre del documento en el encabezado izquierdo
\fancyhead[C]{}  % Centro vacío
\fancyhead[R]{\thepage}  % Numeración de páginas en el encabezado derecho
% \renewcommand{\thepage}{\arabic{page}}  % Asegura la numeración de páginas en números arábigos

% Definición de los anexos
\newcommand{\annex}[1]{\section{Anexo #1} \addcontentsline{toc}{section}{Anexo #1}}
\addto\captionsspanish{%
  \renewcommand{\tablename}{Tabla}%
  \renewcommand{\listtablename}{Índice de tablas}%
}
\title{{ \qrcode[height=1.3cm]{https://doctec.blog/} \\ \small Ref.:\uppercase{Aep030}}\\{\textbf{OPTIMIZACIÓN DE POTENCIA}}}
\author{
pp.ParticipantesP1NombreRaznSocial
}
\date{\today}










\begin{document}


\begin{Form}
	\maketitle

	\tableofcontents  % Tabla de contenidos
	\listoffigures    % Lista de graficos
	\listoftables     % Lista de tablas



    









\section{Introduccióvvn}
Este informe tiene como objetivo analizar y optimizar la potencia contratada en la factura eléctrica de una empresa dedicada a la fabricación de mallas electrosoldadas. La empresa actualmente tiene contratada una potencia de 200 kW en todos los periodos de la tarifa 6.1TD, lo que genera costes elevados debido a excesos de potencia. El objetivo es ajustar la potencia contratada para minimizar los costes anuales.

\section{Metodología}
Para optimizar la potencia contratada, se siguieron los siguientes pasos:

\begin{enumerate}
    \item \textbf{Recopilación de datos}: Se utilizó la curva de carga de consumos horarios del año 2022 y los costes asociados a la tarifa de acceso y excesos de potencia publicados en enero de 2022.
    \item \textbf{Análisis de la tarifa}: Se identificaron los costes del término de potencia y los excesos de potencia para la tarifa 6.1TD.
    \item \textbf{Uso de herramientas de optimización}: Se empleó una hoja de cálculo con la función Solver para determinar la potencia óptima de contratación en cada periodo.
\end{enumerate}

\section{Resultados}
Tras el análisis, se obtuvieron los siguientes resultados:



El coste total anual con la potencia optimizada es de \textbf{20,704.5 €}, lo que representa un ahorro de aproximadamente \textbf{13,000 €} al año en comparación con la configuración anterior.

\section{Conclusiones}
\begin{itemize}
    \item La optimización de la potencia contratada permite reducir significativamente los costes energéticos.
    \item La herramienta Solver proporciona una solución aproximada pero efectiva para determinar la potencia óptima.
    \item Es fundamental analizar periódicamente la curva de consumo y ajustar la potencia contratada para evitar excesos y minimizar costes.
\end{itemize}

\section{Recomendaciones}
\begin{itemize}
    \item Implementar un sistema de monitorización continua del consumo para ajustar la potencia contratada de manera dinámica.
    \item Realizar este análisis anualmente o cuando haya cambios significativos en el consumo energético de la empresa.
    \item Considerar otras medidas de eficiencia energética para complementar la optimización de la potencia contratada.
\end{itemize}
















% --- 7. REFERENCIAS BIBLIOGRÁFICAS ---
\begin{thebibliography}{99}
    \bibitem{ccee} \href{https://www.boe.es/buscar/pdf/2021/BOE-A-2021-7120-consolidado.pdf}{Resolución de 28 de abril de 2021, de la Dirección General de Política Energética y Minas, por la que se establece el contenido
mínimo y el modelo de factura de electricidad a utilizar por los comercializadores de referencia.}


\bibitem{rd244}
\href{}{RD 244/2019  sobre autoconsumo}

\bibitem{esios} 
\href{https://www.esios.ree.es/es/pvpc}{ESIOS - Red Eléctrica de España. PVPC y datos del sistema eléctrico
}


\bibitem{boe}
\href{}{Real Decreto 216/2014 por el que se establece la metodología de cálculo de los precios voluntarios para el pequeño consumidor.}


\bibitem{boe}
\href{https://www.cnmc.es/sites/default/files/3416756_135.pdf}{ACUERDO POR EL QUE SE CONTESTAN CONSULTAS RELATIVAS A LA
APLICACIÓN DE LA CIRCULAR 3/2020, DE 15 DE ENERO, POR LA QUE SE
ESTABLECE LA METODOLOGÍA PARA EL CÁLCULO DE LOS PEAJES DE
TRANSPORTE Y DISTRIBUCIÓN DE ENERGÍA ELÉCTRICA.}




\end{thebibliography}








\ifdefined\MostrarVariablesAlFinal
\newpage
\onecolumn



uu. dfPar

\begin{figure}[H] \centering
                    
                    \includegraphics[width=.5\textwidth]{/tmp/tmpq9tnp93m.png}
                    \caption{Par}
                    
                    \label{fig:dfassssdfsa}
                    \end{figure}
                    

uu. dfSerie1

\begin{figure}[H] \centering
                    
                    \includegraphics[width=.5\textwidth]{/tmp/tmpjx7l7qhc.png}
                    \caption{Serie1}
                    
                    \label{fig:dfassssdfsa}
                    \end{figure}
                    

uu. d10Generacion

\begin{figure}[H] \centering
                    
                    \includegraphics[width=.5\textwidth]{/tmp/tmpx13r3ozi.png}
                    \caption{Generacion}
                    
                    \label{fig:dfassssdfsa}
                    \end{figure}
                    

uu. d11GridConnected

\begin{figure}[H] \centering
                    
                    \includegraphics[width=.5\textwidth]{/tmp/tmp3tk54lmi.png}
                    \caption{Grid connected}
                    
                    \label{fig:dfassssdfsa}
                    \end{figure}
                    

uu. tabladfPar


                    \begin{table}[H] \centering
                        % \resizebox{0.4\textwidth}{!}
                        {
                        \begin{tabular}{ll}
\toprule
 & eee \\
 &  \\
\midrule
saD & DASD \\
aa & 3 \\
bb & rqwer \\
\bottomrule
\end{tabular}

                        }
                        \caption{Par}
                    \end{table}
                    

xx. dfPar

       eee
          
saD   DASD
aa       3
bb   rqwer

xx. dfSerie1

                  kWh
Unnamed: 0           
2022-01-01 0:0:0   13
2022-01-01 1:0:0   14
NaN                13
NaN                14
NaN                13
...               ...
NaN                12
NaN                13
NaN                12
NaN                12
NaN                13

[8760 rows x 1 columns]

xx. d10Generacion

                       P
kWh_hora                
2024-01-01 00:00:00  0.0
2024-01-01 01:00:00  0.0
2024-01-01 02:00:00  0.0
2024-01-01 03:00:00  0.0
2024-01-01 04:00:00  0.0
...                  ...
2024-12-31 19:00:00  0.0
2024-12-31 20:00:00  0.0
2024-12-31 21:00:00  0.0
2024-12-31 22:00:00  0.0
2024-12-31 23:00:00  0.0

[8784 rows x 1 columns]

xx. d11GridConnected

0      59.34
1      76.59
2     114.63
3     138.50
4     167.11
5     177.06
6     189.44
7     167.37
8     127.58
9      92.78
10     61.00
11     51.87
Name: E_m, dtype: float64
\fi
\end{Form}
\end{document}

